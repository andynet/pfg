\begin{table*}
    \label{tab:segment}
    \begin{tabular}{rllll}
        \toprule
        \#sequences & pggb & vg & minigraph-cactus & PFG \\
        \midrule
        % 2 &
        4   & 0m12s & 0m1s  & 3m47s  & 0m1s \\
        16  & 0m19s & 0m1s  & 6m30s  & 0m1s \\
        64  & 1m20s & 0m2s  & 13m30s & 0m1s \\
        256 & 7m20s & 0m55s & 41m49s & 0m3s \\
        \bottomrule
    \end{tabular}
    \caption{
        Comparison of running times of pangenomic graph construction tools.
        Some of these tools need some kind of preprocessing in order to build the graph.
        pggb requires for the fasta files to be indexed.
        In our case, due to the small size of our files, this time was negligible, but it can have a bigger impact when bigger genomes are considered.
        VG builds a pangenome graph starting from a VCF file or a multiple sequence alignment.
        In each case, those outputs require to perform time consuming steps.
        Having a suitable VCF file requires to align each haplotype to a reference genome and subsequently performing variant calling.
        Beyond being time consuming, using a linear reference in a first place can introduce some biases in the first place.
        On the other hand, performing a multiple sequence alignment is also known to be potentially slow, according to the length and the abundance of the underlying sequences.
    }
\end{table*}

