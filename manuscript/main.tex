%% The first command in your LaTeX source must be the \documentclass command.
%%
%% Options:
%% twocolumn : Two column layout.
%% hf: enable header and footer.
\documentclass[twocolumn]{ceurart}

%%
%% One can fix some overfulls
\sloppy

%%
%% Minted listings support 
%% Need pygment <http://pygments.org/> <http://pypi.python.org/pypi/Pygments>
\usepackage{minted}
%% auto break lines
\setminted{breaklines=true}

%%
%% end of the preamble, start of the body of the document source.
\begin{document}

%%
%% Rights management information.
%% CC-BY is default license.
\copyrightyear{2022}
\copyrightclause{
    Copyright for this paper by its authors.
    Use permitted under Creative Commons License Attribution 4.0 
    International (CC BY 4.0).
}

%%
%% This command is for the conference information
\conference{
    Woodstock'22: Symposium on the irreproducible science,
    June 07--11, 2022, Woodstock, NY
}

%%
%% The "title" command
\title{Prefix-free graphs and suffix array construction in sublinear space}

\tnotemark[1]
\tnotetext[1]{
    You can use this document as the template for preparing your
    publication. We recommend using the latest version of the ceurart style.
}

%%
%% The "author" command and its associated commands are used to define
%% the authors and their affiliations.
\author[1,2]{Andrej Baláž}[%
orcid=0000-0002-0877-7063,
email=kulyabov-ds@rudn.ru,
url=https://yamadharma.github.io/,
]
\cormark[1]
\fnmark[1]
\address[1]{
    Peoples' Friendship University of Russia (RUDN University),
    6 Miklukho-Maklaya St, Moscow, 117198, Russian Federation
}

%% Footnotes
\cortext[1]{Corresponding author.}
\fntext[1]{These authors contributed equally.}

%%
%% The abstract is a short summary of the work to be presented in the
%% article.
\begin{abstract}
  Recently, many graph structures for computational pangenomics were proposed.
  In order to be useful, these graph structures need to implement several
  operations such as efficient construction from many complete genomes,
  mapping of short and long read.
  These basic bioinformatics operations are well studied on sequential data,
  and together with data structures such as suffix trees, suffix arrays and 
  burrows wheeler transform allow high performance.
  Attempts to implement these operations on graphs brings with it many complications
  since these date structures are not easily obtainable for graphs.
  In this work, we introduce prefix-free graphs, a pangenomic data structure
  allowing to obtain the well known data structures from stringology in graph
  settings, which in turn allows for many efficient operations on pangenomes.
\end{abstract}

%%
%% Keywords. The author(s) should pick words that accurately describe
%% the work being presented. Separate the keywords with commas.
\begin{keywords}
  LaTeX class \sep
  paper template \sep
  paper formatting \sep
  CEUR-WS
\end{keywords}

%%
%% This command processes the author and affiliation and title
%% information and builds the first part of the formatted document.
\maketitle

\section{Introduction}
Suffix array is a data structure used in many fields, such as stringology,
bioinformatics, and... .
Particularly in bioinformatics, it is used as part of FM index, which in turn
is used for read mapping.
Recently, a new paradigm in bioinformatics was formed, moving from a single
reference genome to a pangenome.
The term pangenome was first used by Tettelin et al. in 2005 during the
resequencing experiments on streptococcus.
Since then, the definition of a pangenome has shifted to any set of sequences
analysed jointly.
Nevertheless, most pangenomes consist of highly related sequences and thus are
highly repetitive.
This leads to representation of pangenomes as graphs, where similar genome
regions are represented by a single node and the nodes are conected to paths
representing the original sequences.
Several approaches of pangenomic graph construction exist, such as variation
graphs, cactus graphs, ...
Most of these approaches require an initial local alignment of similar regions,
which makes them computationally expensive.
Here we present a new class of graphs, prefix-free graphs, which are easy to
construct and provide a way how to efficiently index them to allow pattern 
search queries.
Pangenomes introduced by...
In computational pangenomics...
Repetitive datasets...
Graph pangenomes and textual pangenomes...
Prefix-free parsing is a technique...
Prefix-free graphs play an important role in improving space
complexity of several efficient algorithms, but are never the concept is never
framed as a pangenomic data structure.
We think that separating and clearly defining this data structure can lead to
more advancements...
Our contribution is in defining a data structure called prefix-free graphs and
showing how this data structure can be used to generate suffix array in
sublinear space.
Furthermore, we implemented two binarieš for an easy construction of prefix-free
graphs from a set of sequences, and from a graph in gfa format.
Furthermore, we implemented rust library for working with prefix-free graphs.
This library contains an iterator, which can be directly used to generate the 
suffix array in sublinear space.

\section{Prefix-free graphs}
The idea of prefix-free graphs is inspired by 

\section{Suffix array construction}
\section{Related work}
MARIA \cite{2022maria}
Prefix-free Wheeler graphs \cite{2022pfwg}
Prefix-free parsing for building big BWTs \cite{2019boucher}
Rpair: rescaling repair with rsync \cite{2019gagie}

\section{Experiments}
\section{Discussion}

%%
%% The acknowledgments section is defined using the "acknowledgments" environment
%% (and NOT an unnumbered section). This ensures the proper
%% identification of the section in the article metadata, and the
%% consistent spelling of the heading.
\begin{acknowledgments}
  Thanks to the developers of ACM consolidated LaTeX styles
  \url{https://github.com/borisveytsman/acmart} and to the developers
  of Elsevier updated \LaTeX{} templates
  \url{https://www.ctan.org/tex-archive/macros/latex/contrib/els-cas-templates}.
\end{acknowledgments}

%%
%% Define the bibliography file to be used
\bibliography{sources}

%%
%% If your work has an appendix, this is the place to put it.
\appendix

\section{Online Resources}

The sources for the ceur-art style are available via
\begin{itemize}
\item \href{https://github.com/yamadharma/ceurart}{GitHub},
% \item \href{https://www.overleaf.com/project/5e76702c4acae70001d3bc87}{Overleaf},
\item
  \href{https://www.overleaf.com/latex/templates/template-for-submissions-to-ceur-workshop-proceedings-ceur-ws-dot-org/pkfscdkgkhcq}{Overleaf
    template}.
\end{itemize}

\end{document}

