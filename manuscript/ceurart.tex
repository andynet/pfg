%% Options:

%% twocolumn : Two column layout.
%% hf: enable header and footer.
\documentclass[twocolumn]{ceurart}

%%
%% One can fix some overfulls
\sloppy

%%
%% Minted listings support 
%% Need pygment <http://pygments.org/> <http://pypi.python.org/pypi/Pygments>
\usepackage[finalizecache]{minted}
%% \usepackage[frozencache]{minted}
\usepackage{lmodern}
\usepackage{caption} %% corrects hyperlinks to caption
%% auto break lines
\setminted{breaklines=true}

\begin{document}

\copyrightyear{2023}
\copyrightclause{
    Copyright for this paper by its authors.
    Use permitted under Creative Commons License Attribution 4.0 
    International (CC BY 4.0).
}

\conference{
    ITAT'23: Information technologies -- Applications and Theory
    September 22--26, 2023, Tatranské Matliare, Vysoké Tatry
}

\title{Prefix-free graphs and suffix array construction in sublinear space}

\author[1]{Andrej Baláž}[
    email=andrejbalaz001@gmail.com,
]
\cormark[1]

\author[1]{Alessia Petescia}[]

\address[1]{
    Department of Applied Informatics,
    Faculty of Mathematics, Physics and Informatics,
    Comenius University, Bratislava, Slovakia
}

% Footnotes
\cortext[1]{Corresponding author}

%%
%% The abstract is a short summary of the work to be presented in the
%% article.
\begin{abstract}
A recent paradigm shift in bioinformatics from a single reference genome to a pangenome brought with it several graph structures.
These graph structures must implement operations, such as efficient construction from multiple genomes and read mapping.
Read mapping is a well-studied problem in sequential data, and, together with data structures such as suffix array and Burrows-Wheeler transform, allows for efficient computation.
Attempts to achieve comparatively high performance on graphs bring many complications since the common data structures on strings are not easily obtainable for graphs.
In this work, we introduce prefix-free graphs, a novel pangenomic data structure; we show how to construct them and how to use them to obtain well-known data structures from stringology in sublinear space, allowing for many efficient operations on pangenomes.
\end{abstract}

%%
%% Keywords. The author(s) should pick words that accurately describe
%% the work being presented. Separate the keywords with commas.
\begin{keywords}
    computational pangenomics \sep
    graph pangenome \sep
    suffix array
\end{keywords}

%%
%% This command processes the author and affiliation and title
%% information and builds the first part of the formatted document.
\maketitle

\input{content/01_introduction.tex}
\input{content/02_pfg.tex}
\section{Suffix array construction}
A suffix array is a permutation of string positions which lexicographically sorts the suffixes of the string starting at that position.
For a set of strings, we consider a concatenation of the strings as shown in Figure \ref{fig:path_join}.
In the following text, we will refer to this concatenation as the pangenome.

The suffix array is an influential data structure with many applications in efficient string algorithms solving problems such as exact pattern matching, repeat finding, maximum exact match (MEM) finding, document retrieval and many more.
There exist several algorithms for suffix array construction in linear time \cite{skew2003karkkainen,sais2009nong,saoverview2007puglisi} with several practical implementations \cite{louza2017inducing,libdivsufsort}.
Despite their linear time complexity, these algorithms become bottlenecks in some applications because of their linear space complexity.
This observation is especially relevant in pangenomics, where the datasets often do not fit in the computer memory.

Here, we show another crucial advantage of prefix-free graphs.
Although they do not offer any improvement of theoretical guarantees in the worst case, in practice, they often represent the pangenome in a substantially smaller space and allow us to generate the suffix array values one by one, possibly using the values directly in subsequential computation or storing them in compressed form.
This iteration can be done without ever expanding the pangenome to its full textual representation in space proportional to the sum of segment lengths and the sum of path lengths.

\subsection{Iterator preparation}
To prepare the iterator of a suffix array of the pangenome from a prefix-free graph, we need to create several data structures.
First, we concatenate all the segments into a single string using a separator $\texttt{\#}$ and append a sentinel $\texttt{\$}$.
We will call this concatenation \emph{segment join}.
An example of a segment join is in Figure \ref{fig:ids_and_positions}.

Next, we calculate the segment join's suffix array and the longest common prefix array \cite{landau2001linear,manzini2004two}.
For both of these arrays, there exist algorithms with linear time and space complexity which we can use.
We note that these linear complexities are proportional to the length of the segment join, which is usually much smaller than the original pangenome.

Next, for each suffix of the segment join, we need to calculate the corresponding segment ID and position values.
The value segment ID represents in what segment the current suffix starts, and the value segment position represents at what position in that particular segment the current suffix starts.
These arrays can be computed using an inverse permutation of a suffix array $\texttt{ISA}$ (Equation \ref{eq:isa}) of a segment join in linear time.

\begin{equation}
    \label{eq:isa}
    \texttt{ISA}[\texttt{SA}[i]] = i
\end{equation}

To illustrate the procedure, consider the segment join of our running example from Figure \ref{fig:ids_and_positions}.
Each position of the join can be assigned a segment ID and a position in the current segment by linearly scanning the segment join and incrementing the ID and position accordingly.
Then, applying the ISA to these arrays changes the order of computed values in correspondence to the sorted suffixes.
The resulting suffix array $\texttt{(SA)}$, longest common prefix array $\texttt{(LCP)}$, segment ID array $\texttt{(ID)}$ and segment positions array $\texttt{(pos)}$ are stored in a \emph{suffix table} as shown in Table \ref{tab:suffix}.

\begin{figure}
    \centering
    \includegraphics[width=\linewidth]{images/ids_and_positions.png}
    \caption{
    Segment join of segments from the running example, corresponding segment IDs, and positions. Permuting IDs and positions according to the \texttt{ISA} results in the segment ID array \texttt{ID} and the segment position array \texttt{pos} shown in Table \ref{tab:suffix}. \label{fig:ids_and_positions}
    }
\end{figure}

\input{content/table_suffix.tex}

\begin{table}
    \caption{
    Segment table containing an ID, length, starting positions in the pangenome and ranks of the right contexts of these positions for each segment. \label{tab:segment}}
    \begin{tabular}{rrll}
        \toprule
        id & length & starts & right context ranks \\
        \midrule
        0 & 4 & 9           & 9         \\
        1 & 3 & 15, 1       & 13, 14    \\
        2 & 5 & 18, 5, 11   & 1, 2, 3   \\
        3 & 3 & 8, 14, 0    & 4, 5, 6   \\
        4 & 4 & 16          & 7         \\
        5 & 5 & 2           & 8         \\
        \bottomrule
    \end{tabular}
\end{table}



\begin{figure}
    \centering
    \includegraphics[width=\linewidth]{images/path_join.png}
    \caption{
    Path join of paths from the running example and the corresponding starts and ranks. The concatenated pangenome and segments are shown above to clarify the meaning of the start values. \label{fig:path_join}
    }
\end{figure}

In the suffix table, one row can represent multiple positions of the pangenome.
To identify these positions, we store some additional information in a \emph{segment table}.
For each segment of the prefix-free graph, we store its length, starting positions in the pangenome and ranks of the right contexts of these positions.
To calculate the starting positions and the ranks of the right contexts, we use a \emph{path join}.
Similarly to segment join, a path join is a concatenation with delimiters $\texttt{\#}$ and sentinel $\texttt{\$}$, but now constructed by concatenating the paths.
An example of a path join for our running example is in Figure \ref{fig:path_join}.
Then, starting positions can be calculated by cumulatively summing the lengths of the segments in path join and subtracting the overlaps.

The computation of ranks is more involved.
It uses the normalized form of prefix-free graphs since it relies on a lexicographically smaller ID in a path representing a lexicographically smaller segment.
We construct the suffix array of the path join and find its inverse permutation $\texttt{ISA}$.
$\texttt{ISA}$ gives us ranks for each position in the path join.
To determine the rank of the right context for position $i$, we take
the value of $\texttt{ISA}[i+1]$.
Finally, we store the starts and ranks sorted by the rank values in the segment table as shown in Table \ref{tab:segment}.

\subsection{Iteration}
With the previous tables stored in memory, we have all the necessary ingredients to generate the suffix array value by value.

Each row in the suffix table represents a suffix of a particular segment.
There are four cases of what the first position of these suffixes can represent within the segments:
\begin{itemize}
    \item the sentinel $\texttt{\$}$
    \item a separator $\texttt{\#}$
    \item a position within the last $k$ characters of a segment
    \item a position outside the last $k$ characters, separator and sentinel
\end{itemize}

Since the pangenome has no corresponding position for the sentinel or separator characters, we can skip the first rows representing them.

In the third case, the position is inside the trigger word or the sentinels appended during the graph construction.
The positions inside the trigger words are represented twice in the suffix table, once at the end of a segment and a second time at the beginning of the following segment in the pangenome.
These ending positions can violate the prefix-free property of the segment suffixes and, therefore, can be sorted incorrectly.
Skipping through these positions ensures the prefix-free property for the rest of the suffixes and also avoids double reporting.
Therefore, if the length of a current segment suffix is smaller or equal to the size of the trigger words $k$, we skip the row as in the previous cases.
This choice also plays nicely with the previous choice of appending $k$ sentinels during the construction of a prefix-free graph, as these positions will not get reported either.

Finally, in the last case, we report the suffix array values.
The suffix table can be partitioned into blocks of the same segment suffixes.
For example, consider rows 20 and 21, which form a single block.
All other blocks in the running example consist of single rows; therefore, we call them singletons.

This partitioning leads to three cases:
\begin{itemize}
    \item a singleton block with segment suffix occurring only once in the whole pangenome
    \item a singleton block with segment suffix occurring several times in the pangenome
    \item a non-singleton block
\end{itemize}

In the first case, we must report only a single suffix array value.
Given the row index $i$, this value can be calculated with Equation \ref{eq:sa_value}.

\begin{equation}
    \label{eq:sa_value}
    \texttt{SA value} = \texttt{starts}[\texttt{ID}[i]] + \texttt{pos}[i]
\end{equation}

As an example, consider the row $13$ in Table \ref{tab:suffix}, the first row yielding a SA value.
Its segment ID is $0$, and from Table \ref{tab:segment}, we see only one occurrence of segment $0$ with starting position $9$ in the pangenome.
The offset from the start of a segment $\texttt{pos}[13]$ is $0$.
Summing these two values, we get the first value of a suffix array $9 + 0 = 9$ corresponding to the lexicographically smallest suffix of a pangenome $P[9..] = \texttt{ACACT}$.

The second, slightly more complex case is a singleton block representing a segment suffix with several occurrences in the pangenome.
In this case, we must report as many suffix array values as the number of occurrences.
Because the starting segment positions in the segment table are sorted based on their right context rank, we can iterate through these starting positions and apply Equation \ref{eq:sa_value} to each of them.

As an example, consider the row $14$ in Table \ref{tab:suffix}.
This suffix occurs twice in the pangenome in segments starting at positions $15$ and $1$.
Since the offset from the start of a segment $\texttt{pos}[14]$ is $0$, we report a suffix array values $15 + 0 = 15$ and $1 + 0 = 1$, corresponding to the suffixes $P[15..] = \texttt{ACGACT}$ and $P[1..] = \texttt{ACGTACT}$.

In the last case, we have a non-singleton block representing suffixes of several segments, possibly with multiple occurrences.
These suffixes represent identical substrings in the pangenome.
Here, we report a suffix array value for each of the substrings.
To identify the first value, we must find the starting position with the smallest right context rank.
Because the ranks are sorted, this procedure is similar to the merging phase of a merge sort.
Therefore, to iterate through all suffix array values in the block, we always identify the segment start with the next smallest right context rank and apply Equation \ref{eq:sa_value} to this segment start.

As an example, consider the block of rows $[20..21]$ in Table \ref{tab:suffix}.
The relevant segment IDs are $0$ and $3$, with segment starts at positions $9$, $8$, $14$ and $0$.
The right context ranks from smallest to highest are $4, 5, 6, 9$ with the corresponding segment starts $8, 14, 0, 9$.
Applying Equation \ref{eq:sa_value} to these segment starts yields a suffix array values $8 + 0 = 8$, $14 + 0 = 14$, $0 + 0 = 0$ and $9 + 1 = 10$, representing pangenome suffixes $P[8..] = \texttt{CACACT}$, $P[14..] = \texttt{CACGACT}$, $P[0..] = \texttt{CACGTACT}$, and $P[10..] = \texttt{CACT}$.


\section{Results}
We implemented the prefix-free graphs as a Rust package.
This package contains two binary crates and one library crate.
Binary crates are executables that serve the purpose of creating prefix-free graphs from FASTA and GFA formats.
The binary crates are named \texttt{fasta2pfg} and \texttt{gfa2pfg}, and their usage is as follows:

\begin{minted}{bash}
fasta2pfg -t triggers.txt < pangenome.fna > pfg.gfa

gfa2pfg -t triggers.txt < pangenome.gfa > pfg.gfa
\end{minted}

We compared the running time of the construction algorithm with several pangenome construction tools, namely with the PanGenome Graph Builder \cite{garrison2023building}, VG \cite{garrison2018variation}, and Minigraph-Cactus \cite{hickey2023pangenome}.
For comparison, we used up to 256 SARS-CoV-2 sequences.
These sequences have lengths of around 30 kbp and high nucleotide similarity.
We used stop codons (\texttt{TAA}, \texttt{TAG}, and \texttt{TGA}) as trigger words to construct prefix-free graphs.
The resulting running times are shown in Table \ref{tab:comparison}.

\begin{table*}
    \begin{tabular}{rrrrr}
        \toprule
        \#seqs & pggb & vg & minigraph-cactus & pfg \\
        \midrule
        4   & 0m 12s & 0m 1s  & 3m 47s  & 0m 1s \\
        16  & 0m 19s & 0m 1s  & 6m 30s  & 0m 1s \\
        64  & 1m 20s & 0m 2s  & 13m 30s & 0m 1s \\
        256 & 7m 20s & 0m 55s & 41m 49s & 0m 3s \\
        \bottomrule
    \end{tabular}
    \caption{
        Comparison of running times of pangenomic graph construction tools.
        Wall clock times are measured by the built-in bash command \texttt{time} and rounded up to the nearest second.
        In addition to the measured times, \texttt{pggb} and \texttt{vg} need preprocessing of the pangenome.
        \texttt{pggb} requires the fasta file to be indexed.
        This time was negligible in our case, but it can have a more significant impact when a more extensive dataset is considered.
        \texttt{vg} builds a pangenome from a VCF file or a multiple sequence alignment.
        In both cases, acquiring this data requires time-consuming steps.
        Furthermore, variant calling using a linear reference can introduce a reference bias.
        \label{tab:comparison}
    }
\end{table*}



The library crate provides an interface for working with prefix-free graphs, mainly an iterator of a suffix array in sublinear space and linear time.
The library can be used from within the Rust programming language as follows:

\begin{minted}{rust}
let pfg = PFG::load("pfg.gfa");

for (i, (sa_i, id_i, pos_i)) in pfg.iter().enumerate() {
    println!("{}\t{}\t{}\t{}", i, sa_i, id_i, pos_i);
}
\end{minted}

In addition to the suffix array value, the iterator provides the segment ID and position values.
These are useful for the computation of several related data structures.
Similarly to \citet{2019boucher}, we can use them to compute the Burrows-Wheeler transform $\texttt{BWT}[i]$ by storing the preceding characters of each segment and then reporting the character at position $\texttt{pos}[i]-1$ of a segment $\texttt{id}[i]$.
This computation allows for space-efficient construction of r-index \cite{gagie2020fully} similar to MONI implementation \cite{rossi2022moni}, space-efficient construction of Wheeler graphs similar in nature to the implementation in \citet{2022pfwg}, and with the use of predecessor queries, space-efficient construction of a tag array introduced in \citet{2022maria}.


\section{Conclusion}
An increasing abundance of available genomic data leads to the need for new data structures to utilize all the contained information to its full potential.
Pangenome graphs have been proposed as a solution capable of representing these datasets, exploiting the repetitiveness of a collection of related genomes for efficient storage while preserving and sometimes even highlighting the underlying variations.
However, this shift in the bioinformatics field from linear sequences to pangenomic graphs brings challenges.

The enormous sizes of pangenomes make the construction of pangenomic graphs nontrivial and often one of the major bottlenecks in the analysis.
Current tools rely on well-known computational steps, such as all-against-all alignment \cite{garrison2023building}, multiple sequence alignment \cite{hickey2023pangenome}, or variant calling \cite{garrison2018variation}.
All these steps are computationally expensive and do not scale well with rapidly growing datasets.
Furthermore, variant calling relies on a linear reference, which can introduce a reference bias.

Here, we introduced the prefix-free graph as a standalone data structure and showed how to build it from currently available pangenomic datasets.
In comparison to the other tools, it offers several crucial advantages.
The construction of prefix-free graphs does not require any alignment or variant calling.
Instead, it avoids these expensive steps by employing a set of trigger words, which split the sequences according to their contexts.
The time complexity of this construction is linear with respect to the input size, and the space complexity is sublinear, proportional only to the size of the resulting data structure.

The novelty of prefix-free graphs brings several possible directions for future research, with the two main directions being the connection of prefix-free graphs to stringology and the choice of trigger words.
In Chapter 3, we showed how to use prefix-free graphs to build a suffix array of a pangenome in sublinear space.
Suffix arrays are at the core of many efficient and well-established string algorithms in bioinformatics, such as read mapping and pattern matching.
We expect this connection will facilitate the development of similar algorithms on pangenomic datasets, supporting the paradigm shift.
This direction will require exploring which additional string algorithms can use prefix-free graphs to improve their applicability to these vast and repetitive datasets.
An exciting attempt may be to map reads to prefix-free graphs in a similar fashion as the popular tool BWA \cite{li2009fast} maps reads to a linear reference.

The second direction, the choice of trigger words, is, so far, mostly undiscovered territory.
Trigger words offer great flexibility in the construction process, so a better understanding of their selection is crucial for prefix-free graphs.
From the computational perspective, a set of trigger words minimizing the size of the resulting graph is desirable since it would allow the analysis of larger datasets.
On the other hand, the construction flexibility could be used to create prefix-free graphs based on biologically significant strings.
In our experiments, we used stop codon sequences as the trigger words.
However, experiments with other motives, such as different binding sites, recombination hotspots, or repetitive elements, could illuminate the possibility of capturing particular biological phenomena with prefix-free graphs.
Moreover, integrating the strand information in the graph construction may be beneficial for further reducing the graph size and capturing biological phenomena such as inversion.
This integration could be achieved by considering both the forward and reverse complement of the trigger words.

We believe the characteristics of prefix-free graphs are highly valueable and will have a significant impact when dealing with massive datasets, moving us closer to the ultimate goal of computational pangenomics.



\section{Online Resources}
The sources for the prefix-free graphs are available via
\begin{itemize}
\item \href{https://github.com/andynet/pfg}{https://github.com/andynet/pfg}.
\end{itemize}

\begin{acknowledgments}
This research was funded by a grant from the European Union’s Horizon 2020 research and innovation programme under the Marie Skłodowska-Curie grant agreement No 956229 (ALPACA) and by a grant from Slovak Research Grant Agency VEGA 1/0538/22.
\end{acknowledgments}

\bibliography{sources}

\end{document}

