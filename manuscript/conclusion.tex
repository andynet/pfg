\section{Conclusion}
We introduced the prefix-free graph as a standalone data structure in pangenomics and showed how to use it to build a suffix array of a pangenome in sublinear space.
There are several possible directions for future research on prefix-free graphs.
The first direction may be to answer the question of which additional stringological algorithms and data structures can use prefix-free graphs to improve their applicability to such vast and repetitive datasets as pangenomic data.
Next, the choice of trigger words is, so far, mostly undiscovered territory.
From a computational perspective, the set of trigger words minimizing the size of a resulting prefix-free graph is desirable since it would allow the analysis of larger datasets.
Furthermore, trigger words offer great flexibility in the construction process.
We hypothesize this flexibility could be used for the construction of prefix-free graphs based on biologically significant strings.
In our experiments, we used the stop codon sequences as the trigger words, but other motives, such as different binding sites, recombination hotspots, or repetitive elements, are possible and could be used to capture particular biological phenomena.

