\section{Conclusion}
\subsection{Application}
There are several proposed data structures implicitly relying on the structure
of prefix-free graphs.

Rpair: rescaling repair with rsync \cite{2019gagie}
Prefix-free parsing for building big BWTs \cite{2019boucher}
MONI and r-index \cite{rossi2022moni,gagie2020fully}
Prefix-free Wheeler graphs \cite{2022pfwg}
MARIA \cite{2022maria}

All of these works implement their own version of prefix-free graphs.
We argue that separating prefix-free graphs as a standalone data structure can
bring several benefits:

\begin{itemize}
    \item reduce the complexity of the presentation
    \item allowing for better optimization of algorithms related to prefix-free graphs
    \item supporting the theoretical research by clearly deliminating the relevant terms
    \item allowing to bring prefix-free graphs closer to the biological data
\end{itemize}

Related to the last point, we remark that a possible future research can be in
the proper choice of trigger words.
As an example, we used stop codons in our experiments.
This could create prefix-free graphs capturing some biologically relevant phenomena.
Other options are recombination hotspots, highly repetitive elements, different
binding sites and sites with increased breakage.




